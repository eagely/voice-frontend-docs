\renewcommand*\chapterpagestyle{scrheadings}
\chapter{Implementation and Issues}

\section{Spring Boot Prototype}
\subsection{Backend implementatin}
The backend service was initially implemented in Kotlin using Spring Boot \footnote{The Spring Framework \cite{spring}},
a Java-based framework that provides core infrastructure support for web applications.
The service exposed a prototype REST endpoint at http://localhost:8080/process,
which accepted textual input as a preliminary alternative to audio processing. 
The endpoint implemented pattern matching to route weather-related queries to a weather service,
while other queries were processed through the GPT-4 language model.

\subsection{HTTP client in Qt}
The initial prototype consisted of a minimal user interface
implementing three basic components arranged vertically:
a QLineEdit widget for text input,
a QPushButton for submission,
and a QLabel to display the server's response.
The button's \texttt{clicked()} signal was connected to a slot
handling the network communication.
When the submit buton was clicked, the application sent a POST request
to \texttt{http://localhost:8080/process} containing the user's input as plain text,
and asynchronously processed the server's response through Qt's signal-slot mechanism,
updating the label component to display results or errors.
This HTTP-based communication was later replaced with TCP sockets
to accommodate the backend's transition from REST endpoints,
setting the foundation for the eventual audio processing implementation.
