\renewcommand*\chapterpagestyle{scrheadings}
\chapter{Introduction}
This project is a frontend designed mainly to run on a Raspberry Pi with dedicated hardware,
but other Linux/Unix-like systems are supported too.
It is an extension to a voice assistant diploma thesis, focusing on providing a user-friendly
interface to record audio from the user and forward it to the backend for processing.
The project scope is as follows:
\begin{itemize}
    \item Efficient communication with the backend, with the goal of limiting response delays to 1 second.
    \item Playing back responses from the backend, where the user has the option to receive the output either as text or as audio.
    \item Allowing the user to choose between text input, push-to-talk, and voice activation modes.
\end{itemize}

\section{Background and Motivation}
Voice assistants have steadily gained popularity in recent years, yet they are often perceived
as minor enhancements rather than essential tools. This project aims to change that perception
by providing a voice assistant frontend solution that can enhance users' workflows without being
intrusive --- doing what other voice assistants are known to do but also offering more significant
quality-of-life improvements like workspace management and speech-to-text input.

\section{Technical Challenges}
The development of this project is expected to encounter several technical challenges,
including but not limited to:

\begin{itemize}
    \item \textbf{Voice Activation:} Implementing an effective voice activation system necessitates
    continuous listening capabilities. This can be achieved either through a wake-word detection
    mechanism or an advanced language model that can discern commands from regular speech.
    Both approaches require careful consideration of computational efficiency and responsiveness.

    \item \textbf{Security:} The ability to execute arbitrary commands poses a significant security
    risk. It is imperative to implement strict access controls and validation mechanisms to prevent
    unauthorized actions and protect the user from potential threats posed by malicious actors.

    \item \textbf{User Experience:} Ensuring a seamless and intuitive user experience is vital for
    the adoption of the voice assistant. This includes minimizing response times, providing clear
    feedback, and designing an accessible and user-friendly interface.

    \item \textbf{Low Latency:} Ensuring efficient communication with
    the backend is essential to achieve the goal of response delays no longer than 1 second on a Raspberry Pi.
    While most of the heavy lifting is done by the backend, communication protocols still need to be optimized
    to ensure low latency.
\end{itemize}

\section{Choice of Framework}
Several front-end frameworks were taken into consideration when deciding what to use for the project. These include:

\begin{itemize}
    \item Vue.js
    \item Godot
    \item Qt
    \item SDL
    \item Embedded Graphics
\end{itemize}

Initially, \texttt{Vue.js} and \texttt{Godot} were considered due to prior experience with these frameworks.
However, given that the voice assistant frontend would run on a Raspberry Pi where performance is crucial
for ensuring a smooth user experience, these options were ruled out.

While low-level frameworks like \texttt{SDL} and \texttt{Embedded Graphics} were available,
their steep learning curve and limited industry adoption made them impractical choices.

Ultimately, \texttt{Qt} was selected for its robust performance, extensive industry usage,
comprehensive documentation, and abundant learning resources.
