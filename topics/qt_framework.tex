\renewcommand*\chapterpagestyle{scrheadings}
\chapter{Qt Framework}

The Qt framework is a comprehensive toolkit designed for creating cross-platform applications
with a focus on performance and user experience. It is particularly well-suited for projects
that require a responsive and visually appealing interface, such as the voice assistant frontend
described in this documentation. This chapter provides an overview of the key components
and concepts of the Qt framework, including qmake, signals and slots, Qt Creator, and .ui files.

\section{qmake}
qmake is the build system tool used by Qt to manage the compilation and linking of applications.
It simplifies the build process by automatically generating Makefiles based on project files (.pro).
A typical .pro file includes information about the source files, headers, and other resources needed
for the project. Here is an example of a simple .pro file:

\begin{verbatim}
TEMPLATE = app
TARGET = myapp
QT += core gui
\end{verbatim}

The .pro file specifies that the project is an application (TEMPLATE = app), the target executable
is named "myapp", and it uses the Qt core and GUI modules. Additional elements like source files,
headers, and forms can be specified in similar fashion.

\section{Signals and Slots}
One of the most powerful features of Qt is its signals and slots mechanism, which facilitates
communication between objects. Signals are emitted when a particular event occurs, and slots
are functions that respond to these signals. This mechanism allows for a flexible and decoupled design.

The following example demonstrates connecting a button's click signal to a label's text update slot:

\begin{verbatim}
connect(button, &QPushButton::clicked, label, &QLabel::setText);
\end{verbatim}

In this example, the QPushButton's clicked signal is connected to the QLabel's setText slot.
When the button is clicked, the label's text is updated accordingly.

\section{Qt Creator and .ui Files}
Qt Creator is an integrated development environment specifically designed for Qt development.
It provides comprehensive tools for code editing, debugging, and UI design. The integrated UI editor
allows developers to create and arrange widgets using a visual interface, generating .ui files that
represent the UI layout in XML format.

These .ui files can be integrated into applications using the uic (User Interface Compiler)
tool, which converts the .ui files into C++ code during the build process. This generated code
creates and configures the UI elements exactly as designed in Qt Creator's visual editor.

\section{Layouts and Widgets}
Qt provides a wide range of widgets for building user interfaces, including buttons, labels,
text fields, and more. These widgets can be arranged using layout managers, which ensure that
the UI elements are positioned and resized correctly.

The framework includes several types of layout managers: QHBoxLayout, QVBoxLayout, QGridLayout,
and QFormLayout. Each layout manager arranges widgets in a specific way. QHBoxLayout
arranges widgets horizontally, while QVBoxLayout arranges them vertically.
